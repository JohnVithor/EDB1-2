\subsection{Métodos de Complexidade Constante}

Esses métodos são mais simples de analisar uma vez que suas instruções 
geralmente não envolvem \emph{loops}. A maioria dos métodos de complexidade
constante da classe fazem acesso ou modificação de campos da classe (\emph{
getters} e \emph{setters}). 

Alguns métodos tem como finalidade acessar atributos de forma indireta, 
possibilitando tratamento de erros e garantindo que a instrução não altere 
o atributo. Esses métodos em Programação Orientada a Objetos (POO) são 
chamados de getters. Os getters da classe são:

\textsf{getSize}

Retorna a quantidade de nós na árvore.

\textsf{getRoot}

Retorna um ponteiro para o \emph{node} raiz.

\textsf{getHeight}

Retorna a altura máxima da árvore.

Outros métodos constantes possuem uma quantidade de instruções constante. 
Alguns métodos necessitam fazer operação de potenciação, o que poderia aumentar 
a complexidade, porém como as operações são feitas com potências de 2 o 
problema foi resolvido com a operação de deslocamento a esquerda. Os demais 
métodos de complexidade constante são:

\textsf{ehCheia}

Verifica se o numero de nodes da árvore é igual ao esperado com base na altura 
da árvore por meio da fórmula mostrada em aula.

\textsf{ehCompleta}

Verifica se todas as posições para nodes no penúltimo nível da árvore já foram 
ocupadas, isto é o numero de posições livres é zero.
Caso a árvore tenha seja até o nível 2, ela é automaticamente completa.

\textsf{substituir}

Realiza uma troca entre os dados de dois nós, o ideal seria trabalhar com 
ponteiros para o dado, ou trocar os apontadores de cada nó, mudando seus campos 
de \texttt{parent}, \texttt{left}, \texttt{right}, \texttt{l\_cnt}, 
\texttt{r\_cnt e level}, deixando apenas o campo data intocado.

\textsf{atualizaParent}

Dados dois nós, atualiza a informação do pai do primeiro para que a posição de 
filho direito ou esquerdo em que o primeiro nó ocupava seja trocada para o 
segundo nó informado.
Caso o primeiro nó seja a raiz, o segundo nó ganha esse titulo.

\textsf{atualizaNivelENodes}

Decrementa o numero de nós da árvore, incrementa o numero de nós disponíveis no 
nível do nó indicado, e verifica se é necessário apagar a existência desse 
nível, caso o nó indicado seja o ultimo do nível.


%123456789A123456789A123456789A123456789A123456789A123456789A123456789A123456789
