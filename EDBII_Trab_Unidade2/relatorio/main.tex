\documentclass[
	% -- opções da classe memoir --
	article,			% indica que é um artigo acadêmico
	12pt,				% tamanho da fonte
	oneside,			% para impressão apenas no verso. Oposto a twoside
	a4paper,			% tamanho do papel. 
    % -- opções do pacote babel --
	english,			% idioma adicional para hifenização
	brazil,				% o último idioma é o principal do documento
	]{abntex2}

% ---
% Pacotes fundamentais 
% ---
\usepackage{cmap}				% Mapear caracteres especiais no PDF
\usepackage{lmodern}			% Usa a fonte Latin Modern
\usepackage[T1]{fontenc}		% Seleção de códigos de fonte.
\usepackage[utf8]{inputenc}		% Codificação do documento (conversão 
                                % automática dos acentos)
\usepackage{indentfirst}		% Indenta o primeiro parágrafo de cada seção.
\usepackage{nomencl} 			% Lista de símbolos
\usepackage{color}				% Controle das cores
\usepackage{graphicx}			% Inclusão de gráficos
\usepackage{minted}             % Sintaxe do linguagens de programação
\usepackage{xcolor}
% ---

% ---
% Configurações de aparência do PDF final

% alterando o aspecto da cor azul
\definecolor{blue}{RGB}{0,50,100}
\usemintedstyle{colorful}
% ---		

% ---
% Pacotes de citações
% ---
\usepackage[brazilian,hyperpageref]{backref}	 % Paginas com as citações na bibl
\usepackage[abnt-alf]{abntex2cite}	% Citações padrão ABNT
% \usepackage[num]{abntex2cite}
% ---

% ---
% Configurações do pacote backref
% Usado sem a opção hyperpageref de backref
\renewcommand{\backrefpagesname}{Citado na(s) página(s):~}
% Texto padrão antes do número das páginas
\renewcommand{\backref}{}
% Define os textos da citação
%\renewcommand*{\backrefalt}[4]{
%	\ifcase #1 %
%		Nenhuma citação no texto.%
%	\or
%		Citado na página #2.%
%	\else
%		Citado #1 vezes nas páginas #2.%
%	\fi}%
% ---

% ---
% Informações de dados para CAPA e FOLHA DE ROSTO
% ---
\titulo{Implementação de Uma Árvore de Busca Binária com 
Operações Adicionais Otimizadas}
\autor{Gleydvan Macedo \and
João Vítor Venceslau Coelho \and
Josivan Medeiros da Silva Gois}
\local{Brasil}
\data{\today}
\orientador[Docente:]{Silvia Maria Diniz Monteiro Maia}
\instituicao{%
  Universidade Federal do Rio Grande do Norte -- UFRN
  \par
  Instituto Metrópole Digital -- IMD
  \par
  Bacharelado em Tecnologia da Informação}
\tipotrabalho{Tese (Doutorado)}
% O preambulo deve conter o tipo do trabalho, o objetivo, 
% o nome da instituição e a área de concentração 
\preambulo{Relatório referente ao trabalho da segunda unidade da disciplina de 
Estruturas de Dados Básicas.}
% ---

% informações do PDF
\makeatletter
\hypersetup{
     	%pagebackref=true,
		pdftitle={\@title}, 
		pdfauthor={\@author},
    	pdfsubject={Implementação de Uma Árvore de Busca Binária com Operações 
    	Adicionais Otimizadas},
	    pdfcreator={LaTeX with abnTeX2},
		pdfkeywords={ávore}{árvore binária}{árvore binária de busca}{estrutura 
		de dados}{análise de complexidade}, 
		colorlinks=true,       		% false: boxed links; true: colored links
    	linkcolor=blue,          	% color of internal links
    	citecolor=blue,        		% color of links to bibliography
    	filecolor=magenta,      		% color of file links
		urlcolor=blue,
		bookmarksdepth=4
}
\makeatother
% --- 

% ---
% compila o indice
% ---
\makeindex
% ---

% ---
% Altera as margens padrões
% ---
\setlrmarginsandblock{4cm}{4cm}{*}
\setulmarginsandblock{4cm}{4cm}{*}
\checkandfixthelayout
% ---

% --- 
% Espaçamentos entre linhas e parágrafos 
% --- 

% O tamanho do parágrafo é dado por:
\setlength{\parindent}{1.3cm}

% Controle do espaçamento entre um parágrafo e outro:
\setlength{\parskip}{0.2cm}  % tente também \onelineskip

% Espaçamento simples
\SingleSpacing

% ----
% Início do documento
% ----
\begin{document}

% Retira espaço extra obsoleto entre as frases.
\frenchspacing 

% ----------------------------------------------------------
% ELEMENTOS PRÉ-TEXTUAIS
% ----------------------------------------------------------

% Página de Título
% \maketitle
% Folha de Rosto
\imprimirfolhaderosto

% ---
% inserir o sumario
% ---
\pdfbookmark[0]{\contentsname}{toc}
\tableofcontents*
\cleardoublepage
% ---

% ----------------------------------------------------------
% ELEMENTOS TEXTUAIS
% ----------------------------------------------------------
\textual

% ----------------------------------------------------------
% Introdução
% ----------------------------------------------------------
\section*{Introdução}

Este documento apresenta uma discussão sobre a classe de Árvore Binária de 
Busca(ABB) 
implementada para o trabalho da segunda unidade da disciplina de Estruturas de 
Dados Básicas. 
Uma ABB implementa operações de busca, inserção e remoção. 
Além dessas operações básicas a classe criada tem os métodos de acessar o 
enésimo elemento em ordem simétrica, 
retornar o índice de um elemento, determinar mediana, se a árvore é cheia, se é 
completa e uma função para retornar uma representação por nível da árvore em 
uma string.

O objetivo foi de melhorar o desempenho da estrutura para a execução dessas 
operações adicionando, quando necessário, atributos extras. 
A seguir estão listadas os métodos criados e suas respectivas complexidades.

% ----------------------------------------------------------
% Desenvolvimento
% ----------------------------------------------------------

% ----------------------------------------------------------
% Seção de Metodologia
% ----------------------------------------------------------
\section{Metodologia}

A linguagem utilizada foi C++. O tipo dos dados armazenados na estrutura é 
o tipo primitivo $ int $. A implementação da classe ABB (Árvore de Busca 
Binária) foi feita em um arquivo \texttt{.cpp} e a declaração do cabeçalho
dos métodos, bem como os atributos da classe, foram feitos em um arquivo
\texttt{.hpp}. O compilador utilizado foi o g++ 5.4.0. Além dos arquivos da
classe em si, foram criados casos de teste para verificar o funcionamento da
estrutura de dados criada.

Na implementação algumas alterações foram feitas, tanto na \emph{class} ABB
quanto na \emph{struct} como pode ser visto no \autoref{node} e no 
\autoref{abb}.

% ----------------------------------------------------------
% Seção de Análise
% ----------------------------------------------------------
\section{Análise de Complexidade}

Nenhuma das complexidades chegou a ser maior que $ O(n) $, como esperado.
Em alguns casos não foi possível determinar um $ \Theta (n) $ pois a 
complexidade de alguns métodos dependiam da construção da árvore. 

%123456789A123456789A123456789A123456789A123456789A123456789A123456789A123456789


% \nopagebreak[4]
\subsection{Métodos de Complexidade Constante}

Esses métodos são mais simples de analisar uma vez que suas instruções 
geralmente não envolvem \emph{loops}. A maioria dos métodos de complexidade
constante da classe fazem acesso ou modificação de campos da classe (\emph{
getters} e \emph{setters}). 

Alguns métodos tem como finalidade acessar atributos de forma indireta, 
possibilitando tratamento de erros e garantindo que a instrução não altere 
o atributo. Esses métodos em Programação Orientada a Objetos (POO) são 
chamados de getters. Os getters da classe são:

\textsf{getSize}

Retorna a quantidade de nós na árvore.

\textsf{getRoot}

Retorna um ponteiro para o \emph{node} raiz.

\textsf{getHeight}

Retorna a altura máxima da árvore.

Outros métodos constantes possuem uma quantidade de instruções constante. 
Alguns métodos necessitam fazer operação de potenciação, o que poderia aumentar 
a complexidade, porém como as operações são feitas com potências de 2 o 
problema foi resolvido com a operação de deslocamento a esquerda. Os demais 
métodos de complexidade constante são:

\textsf{ehCheia}

Verifica se o numero de nodes da árvore é igual ao esperado com base na altura 
da árvore por meio da fórmula mostrada em aula.

\textsf{ehCompleta}

Verifica se todas as posições para nodes no penúltimo nível da árvore já foram 
ocupadas, isto é o numero de posições livres é zero.
Caso a árvore tenha seja até o nível 2, ela é automaticamente completa.

\textsf{substituir}

Realiza uma troca entre os dados de dois nós, o ideal seria trabalhar com 
ponteiros para o dado, ou trocar os apontadores de cada nó, mudando seus campos 
de \texttt{parent}, \texttt{left}, \texttt{right}, \texttt{l\_cnt}, 
\texttt{r\_cnt e level}, deixando apenas o campo data intocado.

\textsf{atualizaParent}

Dados dois nós, atualiza a informação do pai do primeiro para que a posição de 
filho direito ou esquerdo em que o primeiro nó ocupava seja trocada para o 
segundo nó informado.
Caso o primeiro nó seja a raiz, o segundo nó ganha esse titulo.

\textsf{atualizaNivelENodes}

Decrementa o numero de nós da árvore, incrementa o numero de nós disponíveis no 
nível do nó indicado, e verifica se é necessário apagar a existência desse 
nível, caso o nó indicado seja o ultimo do nível.


%123456789A123456789A123456789A123456789A123456789A123456789A123456789A123456789

% \nopagebreak[4]
%123456789A123456789A123456789A123456789A123456789A123456789A123456789A123456789

\subsection{Métodos de complexidade Linear}

Um algoritmo de complexidade linear é ótimo nas situações em que é preciso 
visitar todos os nós da árvore. Nesses casos o limite inferior do problema 
também é O(n), o que faz do algoritmo assintoticamente ótimo.

\subsubsection{Métodos que Acessam Todos os Nós}

Os métodos que acessam todos os nós são algoritmos $\Theta$(n) e, por tanto,
 são assintoticamente ótimos (já que esse também é o limite inferior do 
 problema).

\textsf{recursiveErase}

Utiliza a ideia do percusso em pós-ordem para deletar todos os nós a partir de 
uma raiz, apagando recursivamente suas sub-árvores. 

\textsf{toString}

Utiliza uma fila para percorrer a árvore por nível, algoritmo baseado no 
pseudo-código mostrado em aula.

\subsubsection{Métodos Dependentes da Altura}

A complexidade desses algoritmos depende da construção da ABB. Quando a 
árvore é completa eles são $O(log n)$ e são $O(n)$ para árvores ziguezague.

\textsf{search}

Utiliza a estratégia padrão para a busca numa árvore binária de busca, 
comparando a partir da raiz com o dado buscado, e caso o dado seja encontrado, 
retorna o nó atual, caso o conteúdo do nó atual seja menor que o buscado, 
procuramos na subárvore a direita, senão na subárvore a esquerda. Se chegarmos 
numa subárvore vazia retornamos \texttt{nullptr}.

\textsf{insere} (ver \autoref{inserir})

Utiliza a mesma estratégia da busca, porém se o dado for encontrado retorna 
false, pois não podemos inserir um dado repetido, e quando chegarmos numa 
subárvore vazia, inserimos o um novo nó com o dado, ao inserirmos verificamos 
em qual nível da árvore esse nó foi criado, caso seja o primeiro nó de um novo 
nível, adicionamos uma nova posição a um vetor que contabiliza o numero de nós 
em cada nível, além de outras informações, durante a volta da recursão, caso o 
novo nó tenha sido inserido, atualizamos os contadores de filhos a esquerda ou 
a direita do nó, de acordo com a posição que o dado foi inserido.

\textsf{minimum}

Dado um nó de inicio avança sempre para o menor nó dessa subárvore, isto é para 
o nó a esquerda. até encontrar uma subárvore vazia, e então retorna o nó 
anterior a ela. Caso o nó fornecido já seja nulo, retorna a raiz da árvore.

\textsf{maximum}

Dado um nó de inicio avança sempre para o maior nó dessa subárvore, isto é para 
o nó a direita. até encontrar uma subárvore vazia, e então retorna o nó 
anterior a ela. Caso o nó fornecido já seja nulo, retorna a raiz da árvore.

\textsf{remove} (ver \autoref{remover})

Utiliza a mesma ideia de buscar um elemento e, caso ele seja encontrado, o 
elemento é removido da árvore. A remoção possui diferentes passos dependendo do 
número de filhos do nó a ser removido.\\
Caso não possua filhos o nó apenas é removido e os devidos dados atualizados.
O número de nós a esquerda ou a direita dos nós acima deste são atualizados na 
volta da recursão.\\
Caso tenha apenas um filho, o nó filho é colocado no lugar do nó removido, e 
então as informações são atualizadas.\\
Caso tenha dois filhos, então é buscado o substituto do nó a ser removido 
utilizando o método \texttt{\_minimum\_} no filho a direita, e então a nova 
posição do nó é removida, e então os dados atualizados.\\
Se tomarmos a altura do nó a ser removido como sendo \textsf{k}, e sabendo que 
o \texttt{\_minimum\_} será chamado para uma subárvore de \textsf{k}, podemos 
concluir que o número de nós percorrido pela chamada do \texttt{\_minimum\_} 
será menor ou igual a: $ h - k - 1 $. Onde $ h $ é a altura máxima da árvore.

\textsf{enesimoElemento}

Utiliza uma ideia similar a busca, porém em vez de usar o campo data, utiliza o 
número de filhos a esquerda de um \emph{node}. O índice de um \emph{node} é o 
número de \emph{nodes} a esquerda dele(\texttt{l\_cnt}) mais um. Caso o índice 
seja igual a essa soma retornamos o campo data do node atual, caso o índice seja 
maior, verificamos a subárvore a direita e armazenamos o número de \emph{nodes} 
a esquerda do node atual para somar o total, senão apenas avançamos para a 
subárvore a esquerda, e repetimos esse processo enquanto não encontrarmos uma 
subárvore vazia.

\textsf{posicao}

Utiliza uma ideia muito semelhante ao busca e ao enesimoElemento, porém retorna 
o índice do elemento buscado. Utiliza a estratégia de busca para encontrar o 
elemento e a estratégia do enesimoElemento para descobrir o índice.

\textsf{mediana}

Utiliza a definição da mediana como elemento que divide o conjunto de dados em 
dois, a informação do número de elementos da árvore e o método 
\textsf{enesimoElemento} para descobrir qual o elemento que ocupa a posição 
central (\textsf{enesimoElemento(n/2)} sendo $n$ o número de nós). 

% ---
% Finaliza a parte no bookmark do PDF, para que se inicie o bookmark na raiz
% ---
\bookmarksetup{startatroot}% 
% ---



% ----------------------------------------------------------
% Conclusão
% ----------------------------------------------------------
\section*{Considerações finais}
\addcontentsline{toc}{section}{Considerações finais}

Existem hoje várias estruturas de dados prontas para uso e que atendem diversas
necessidades de problemas recorrentes na computação. Porém, nem sempre 
encontramos nessas estruturas a melhor solução para nossos problemas. Por isso é
importante sempre estudar os problemas, analisar as soluções e fazer as 
alterações necessárias para buscar a medida do possível soluções ótimas.
No caso estudado vimos que a árvore de busca binária precisou ser alterada para 
que pudéssemos reduzir a complexidade, adicionando atributos e alterando 
instruções.
%\begin{citacao}
% Nem tudo que reluz é ouro
 
% Nem todos que vagueiam estão perdidos
 
% O velho que é forte não murcha
 
% O gelo não atinge raízes profundas
%\end{citacao}

\pagebreak
% ----------------------------------------------------------
% ELEMENTOS PÓS-TEXTUAIS
% ----------------------------------------------------------
\postextual

% ----------------------------------------------------------
% Referências bibliográficas
% ----------------------------------------------------------
\nocite{book:339441, book:562370, jayme}
\bibliographystyle{abnt-alf}
\bibliography{referencias}

% ----------------------------------------------------------
% Glossário
% ----------------------------------------------------------
%
% Há diversas soluções prontas para glossário em LaTeX. 
% Consulte o manual do abnTeX2 para obter sugestões.
%
%\glossary

% ----------------------------------------------------------
% Apêndices
% ----------------------------------------------------------
\pagebreak
% ---
% Inicia os apêndices
% ---
\begin{apendicesenv}

% ----------------------------------------------------------
\chapter{\label{node}Cabeçalho do Node}
% ----------------------------------------------------------

\definecolor{BckGnd}{HTML}{FEEEDD}      % Define a cor de fundo do código
\inputminted[
    frame=lines,
    framesep=2mm,
    baselinestretch=1.2,
    bgcolor=BckGnd,
    fontsize=\footnotesize,
    linenos,
    firstline=30,
    lastline=81
]{cpp}{ABB.h}

% ----------------------------------------------------------
\chapter{\label{abb}Cabeçalho da ABB}
% ----------------------------------------------------------

\inputminted[
    frame=lines,
    framesep=2mm,
    baselinestretch=1.2,
    bgcolor=BckGnd,
    fontsize=\footnotesize,
    linenos,
    firstline=83,
    lastline=104
]{cpp}{ABB.h}

% ----------------------------------------------------------
\chapter{\label{inserir}Inserir Elemento}
% ----------------------------------------------------------

\inputminted[
    frame=lines,
    framesep=2mm,
    baselinestretch=1.2,
    bgcolor=BckGnd,
    fontsize=\footnotesize,
    linenos,
    firstline=78,
    lastline=129
]{cpp}{ABB.cpp}

% ----------------------------------------------------------
\chapter{\label{remover}Remover Elemento}
% ----------------------------------------------------------

\inputminted[
    frame=lines,
    framesep=2mm,
    baselinestretch=1.2,
    bgcolor=BckGnd,
    fontsize=\footnotesize,
    linenos,
    firstline=164,
    lastline=215
]{cpp}{ABB.cpp}

\end{apendicesenv}
% ---

% ----------------------------------------------------------
% Anexos
% ----------------------------------------------------------
% \cftinserthook{toc}{AAA}
% ---
% Inicia os anexos (Coisas que não foi você que fez)
% ---
%\anexos
% \begin{anexosenv}

% ---
% \chapter{Título do Anexo}
% ---

%\end{anexosenv}

\end{document}
